\chapter{Introduction} 
\label{chapter:introduction}
There are papers describing blockchain technology as a disruptive innovation \cite{atzori2015blockchain}. The \textit{Harvard Business Review} argues blockchain technology is not a disruptive, but a foundational technology \cite{harvardbusinessreview}. Whether blockchain is a disruptive technology or not, it is largely agreed upon that blockchain technology has immense potential and can revolutionize business and redefine companies and economies. But there are still lots of barriers. Currently and in the last years, blockchain technologies such as Bitcoin \cite{nakamoto2012bitcoin}, Ethereum \cite{buterin2014ethereum}, IOTA \cite{Newcomb:2017:ICI:3133850.3133860} or Hyperledger fabric \cite{Androulaki:2018:HFD:3190508.3190538} are a very hot topic. According to the Gartner Hype Cycle Blockchain technology was undergoing the peak of inflated expectation in 2017 \cite{gartner}. Price and market capitalization changes of cryptocurrencies are widely covered by the media. This shows Blockchain technologies are believed to have a great potential and a big variety of new use cases. 

Blockchain technology enables decentralized consensus and can be used for record keeping. Bitcoin is the first digital currency to solve the double-spending problem without the need of a trusted authority. One of the main problems of Bitcoin or in general of blockchains is the low maximum amount of possible processed transactions per second. Additionally, the Bitcoin community disagrees about how to solve this scalability problem. It already split into multiple communities with different approaches and Bitcoin forks.

VIBES (Visualizations of Interactive, Blockchain, Extended Simulations) is a blockchain simulator, which allows fast, scalable and configurable network simulations on a single computer without any additional resources. It was developed in a master thesis by Lyubomir Stoykov \cite{vibes} and is the foundation of this master thesis.

\section{Motivation} 
\label{sec:motivation}
The implications of blockchain protocol changes on key figures like scalability or security are difficult to predict. The reasons are complex relationships between the different parameter of the blockchain network, which can change the way nodes interact with each other or how blocks are handled. This may lead to getting the impression, the blockchain network is a black box. A configurable simulator can help to answer questions about scalability or security.

The goal of this master thesis is to improve VIBES to make more realistic bitcoin-like blockchain simulations possible. In the future, BBSS could be used by developers or heavy blockchain users to simulate changes to different blockchains, so it can help the bitcoin community to agree on bitcoin improvement proposals (BIP).

\section{Problem Statement} 
\label{sec:problemStatement}
Ideally, the users of VIBES could specify their Bitcoin-like Blockchain Simulation. VIBES then presents the results of the simulation in a very detailed and still easily understandable way.

In reality, VIBES does not yet support Bitcoin-like Blockchain Simulations. The Generic Simulations lack defining traits of Bitcoin like maximum block size, SegWit, transaction incentives or attack scenarios. Also, other important metrics are not yet displayed.

To achieve our goals VIBES needs to be changed in a way to allow multiple strategies in the front- and backend, while still having good maintainability of the code. A Bitcoin-like Blockchain Simulation needs to be implemented next to the existing Generic Simulation.

These extensions can improve the quality of the simulations and the use cases of VIBES. For example, the implications of Segwit2x could be analysed. Maybe these extensions could also make it possible to realistically simulate the current bitcoin blockchain with an as similar as possible configuration.

The focus of this master thesis is to implement a Bitcoin-like Blockchain Simulation with all important features to make the simulations as realistic as possible, to make it possible to simulate attacks like double-spending and to visualise all important outputs.

The approach is evaluated by six key figures: correctness, speed,
scalability, flexibility, extensibility and powerful visuals.

\section{Approach}
 \label{sec:approach}
The presented goals should be achieved while maintaining and extending the design and architecture of the existing VIBES framework, which are described in Chapter \ref{chapter:relatedWork}. The Bitcoin-like Blockchain Simulation fast-forwards the whole network ahead of time and skips heavy computations such as solving a block. The actor model is used to achieve simulation at the event level. The Coordinator or also called MasterActor takes the role of an application-level scheduler to make fast-forwarding possible. The Coordinator and other actors need to be adjusted for the Bitcoin-like Blockchain Simulation and attack scenarios. For the visual side, the pattern of Atomic Design \cite{atomicdesign} is maintained to add new visualisations to a high-quality and composable user interface.

\section{Contribution}
 \label{sec:contribution}
Before BBSS, there was no multi-purpose bitcoin simulator with good visualisations. BBSS makes bitcoin-like blockchains easy to understand and allows the optimization of bitcoin-like blockchains in scalability and security. Different scalability related changes like block size limit, SegWit or transaction incentives can be analysed. BBSS also allows the simulation of attacks like double-spending and transaction spam. The changes were implemented in a way to enable good maintainability to adjust the simulation to future changes of the bitcoin protocol or to add new attacks.

\section{Organization}
 \label{sec:organization}
The structure of the thesis begins with this chapter as an introduction. The second chapter describes the background and the theoretical foundations which are necessary for the understanding of this thesis. The Chapter \ref{chapter:relatedWork}: Related Work presents related research papers and the architecture and design of VIBES on which this thesis builds on. And it also shows other related works and puts this work into perspective. Chapter \ref{chapter:approach} describes the approach and implementation details are explained. Developers or users of the simulator can use Chapter \ref{chapter:approach} as documentation to explain questions about the behaviour of the simulations. Chapter \ref{chapter:evaluation}: Evaluation uses empirical and theoretical analysis to test the implementation according to our predefined criteria. In the last chapter, the status, conclusions and, suggestions for future work are summarised.

